\chapter*{ВВЕДЕНИЕ}
\addcontentsline{toc}{chapter}{ВВЕДЕНИЕ}

% В настояще время все больше внимания привлекает тема автоматизации судоходства, а так же общее повышение уровня безопасмности в процессе эксплуатации судов.

% Методы распознавания различных надводных объектов имеют особое значение в данном контексте, так как позволяют решить множество проблем: избежание столкновений судов, осуществление автономного плавания и пр.~\cite{ship-detection}.

% Не менее важной является задача распознавания малых надводных объектов, поскольку традиционные методы обнаружения, основанные на использовании радара, не подходят для задачи обнаружения близко расположенных и малых объектов~\cite{small-ship-detection}.

% Кроме того, обнаружение активности рыболовецких судов по-прежнему является сложной задачей для многих стран, расположенных на архипелагах, например -- Индонезии. В настоящее время для мониторинга огромной морской акватории используется технология, использующая датчики SAR для обнаружения кораблей, разрабатываемая с 1985 года. Однако стоимость использования данной технологии является одним из препятствий для дальнейшего развития технологии~\cite{boats-recognition}.

Целью настоящей работы является разработка метода распознавания надводных объектов с аэрофотоснимков. Для достижения данной цели требуется решить следующие задачи:
%\begin{itemize}
%    \item[---] описать процесс распознавания объекта с изображения;
%    \item[---] изучить современные подходы к решению данного класса задач;
%    \item[---] описать формальную постановку задачи;
%    \item[---] описать существующие методы.
%\end{itemize}