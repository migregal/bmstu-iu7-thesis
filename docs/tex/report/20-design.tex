\chapter{Конструкторский раздел}

\section{Требования к разрабатываемому методу}

Метод обнаружения надводных объектов должен:
\begin{itemize}[label=---]
    \item принимать на вход изображения в форматах PNG, JPG, JPEG;
    \item TODO
\end{itemize}

\section{Требования к разрабатываемому программному комплексу}

Программный комплекс, реализующий интерфейс для разработанного метода, должен предоставлять:
\begin{itemize}[label=---]
    \item возможность загрузки изображений через графический интерфес;
    \item возможность создания итогового изображения с обрамляющими окнами обнаруженных объектов.
\end{itemize}

\section{Выбор семейства/метода обнаружения}

\section{Проектирование метода обнаружения}

В разрабатываемом методе будет использоваться CNN YOLOv8n.

\section{Структура разрабатываемого программного комплекса}

\section{Данные для обучения модели}

В качестве данных для обучения моделей были выбраны два набора данных:
\begin{itemize}[label=---]
    \item kaggle-ships-in-google-earth~\cite{kaggle-ships-in-google-earth-dfqwt_dataset};
    \item VAIS\_RGB+SMD+MARITIME+WSODD+MARVEL~\cite{vais_rgb-smd-maritime-wsodd-marvel_dataset};
    \item \cite{ship-ubxk4_dataset}.
\end{itemize}

\section{Формат хранения разметки данных}

\section{Обучение и тестирование слабых экспертов}

\section{Вывод}