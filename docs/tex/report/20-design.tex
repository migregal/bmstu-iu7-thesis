\chapter{Конструкторский раздел}

\section{Требования к разрабатываемому методу}

Метод обнаружения надводных объектов должен:
\begin{itemize}[label=---]
    \item принимать на вход изображения в форматах PNG, JPG, JPEG, BMP с размером от $640 \times 640$ до $1280 \times 1280$ включительно;
    \item производить распознавание различимых надводных объектов.
\end{itemize}

\section{Требования к разрабатываемому программному комплексу}

Программный комплекс, реализующий интерфейс для разработанного метода, должен предоставлять:
\begin{itemize}[label=---]
    \item возможность загрузки изображений через графический интерфейс;
    \item возможность создания итогового изображения с обрамляющими окнами обнаруженных объектов.
\end{itemize}

\section{Выбор семейства/метода обнаружения}


В разрабатываемом методе будет использоваться CNN YOLOv8n.

\section{Проектирование метода обнаружения}

\includeimage{method-a1}{f}{h}{0.9\textwidth}{IDEF0-диаграмма. Уровень A1}

\includeimage{method-a2}{f}{h}{0.8\textwidth}{IDEF0-диаграмма. Ветка A2}

\section{Структура разрабатываемого программного комплекса}

Разрабатываемый программный комплекс состоит из двух модулей:
\begin{itemize}[label=---]
    \item модуль, реализующий модель YOLOv8 сети для распознавания объектов;
    \item пользовательское приложение, производящее распознавание объектов на основе полученной модели.
\end{itemize}

\subsection{Модуль YOLOv8 модели}

В данном модуле происходит только тренировка и сохранения модели. Модуль должен быть использован только один раз при первой тренировке модели или когда в модель требуется внести изменения. Вся последующая работа с моделью ведется через файл, который содержит в себе натренированную модель, так как процесс тренировки занимает продолжительное время.

\subsection{Модуль пользовательского приложения}

\section{Данные для обучения модели}

В качестве данных для обучения моделей были выбраны два набора данных:
\begin{itemize}[label=---]
    \item kaggle-ships-in-google-earth~\cite{kaggle-ships-in-google-earth-dfqwt_dataset};
    \item VAIS\_RGB+SMD+MARITIME+WSODD+MARVEL~\cite{vais_rgb-smd-maritime-wsodd-marvel_dataset};
    \item \cite{ship-ubxk4_dataset}.
\end{itemize}

\section{Формат хранения разметки данных}

Формат хранения данных обусловлен соглашением о формате в семействе YOLO.

\section{Обучение и тестирование слабых экспертов}

Так как в данной методе используется бэггинг, то каждый из слабых экспертов будет обучаться независимо от остальных. В связи с этим, выборка должна быть разделена на $n$ непересекающихся выборок, где $n$ --- число слабых экспертов.

Для обучения и тестирования слабых экспертов полученные после разделения выборки примерно разделяются в соотношении $7:2:1$ на обучающую, тестовую и валидационную подвыборки, соответственно.

\section{Вывод}