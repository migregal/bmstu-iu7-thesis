\chapter*{ЗАКЛЮЧЕНИЕ}
\addcontentsline{toc}{chapter}{ЗАКЛЮЧЕНИЕ}

Были рассмотрены виды классификаторов и их применимость при распознавании надводных объектов с аэрофотоснимков.

Было дано определение понятия нейронной сети, описаны виды нейронных сетей и принцип их работы.

Были рассмотрены особенности применения нейронных сетей в качестве слабых экспертов для распознавания объектов по фотоснимкам.

Были рассмотрены и проанализированы технологии (семейства R–CNN и YOLO) для распознавания объектов при помощи нейронных сетей, приведены преимущества и недостатки рассмотренных технологий в поставленной задаче.

Так же, был спроектирован метод распознавания надводных объектов с аэрофотоснимков, а также программный комплекс, реализующий интерфейс для метода. Были выбраны наборы данных для обучения модели, обоснован их выбор.

Был разработан программный комплекс, состоящий из двух модулей: модуль модели YOLOv8 и модуль пользовательского приложения. Приведены результаты обучения модели и показаны примеры использования разработанного программного комплекса.

Было проведено исследование работы разработанного программного комплекса. В результате исследования были выделены достоинства, среди которых универсальность, высокая точность и низкое время ответа, а так же недостатки, среди которых требования к использовуемому кластера.

Таким образом, поставленная цель работы --- разработать метод распознавания надводных объектов с аэрофотоснимков, была достигнута.