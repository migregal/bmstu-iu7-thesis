\chapter{Технологический раздел}

\section{Средства реализации}

\subsection{Выбор языка программирования}

Для написания программного комплекса будет использоваться язык программирования Python~3~\cite{python3}. Данный выбор обусловлен следующими факторами:
\begin{itemize}[label=---]
    \item широкий набор библиотек для работы с нейронными сетями;
    \item возможность тренировать нейронную сеть на графическом процессоре с
использованием технологии CUDA~\cite{cuda}.
\end{itemize}

\subsection{Выбор библиотеки глубокого обучения}

Для создания и обучения модели нейронной сети была выбрана библиотека PyTorch~\cite{pytorch} версии 2.0.0. Выбор данной версии обусловлен поддержкой CUDA 11.8, предоставляемой GPU NVIDIA GeForce RTX 2060~\cite{rtx2060}, на котором будет производиться обучение нейронной сети.

\section{Реализация программного комплекса}

\subsection{Тренировка модели}

\section{Результаты обучения модели}

\section{Примеры использования разработанного программного комплекса}

\section{Вывод}