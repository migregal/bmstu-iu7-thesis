\chapter{Технологический раздел}

\section{Средства реализации}

\subsection*{Выбор языка программирования}

Для реализации программного комплекса будет использоваться язык программирования Python~3~\cite{python3}. Данный выбор обусловлен следующими факторами:
\begin{itemize}[label=---]
    \item широкий набор библиотек для работы с нейронными сетями;
    \item возможность обучать нейронную сеть на графическом процессоре с
использованием технологии CUDA~\cite{cuda}.
\end{itemize}

Кроме того, как указывалось ранее, для выбранного языка программирования уже существуют общедоступные пакеты для работы с выбранным методом обнаружения, включая методы для обучения, тестирования и валидации работы метода.

\subsection*{Выбор библиотеки глубокого обучения}

Для создания и обучения модели нейронной сети была выбрана библиотека PyTorch~\cite{pytorch} версии 2.0.0. Выбор данной версии обусловлен поддержкой CUDA 11.8, предоставляемой GPU NVIDIA GeForce RTX 2060~\cite{rtx2060}, на котором будет производиться обучение нейронной сети.

\section{Реализация программного комплекса}

\subsection{Модуль YOLOv8}

В листинге~\ref{lst:detect.py} приведена реализация модуля модели, отвечающего за обнаружение надводных объектов.

\subsection{Обучение слабых экспертов}

В предоставляемом Ultralytics пакете для работы с YOLOv8 предусмотрены методы для обучения моделей с различными параметрами и возможностью применения таких оптимизаций, как батчинг изображений, мозаичная аугментация на ранних стадиях обучения, отслеживание результатов процесса обучения с прерыванием в случае простоя и прочие~\cite{pkg-ultralytics}.

Данные методы
На рисунках~\ref{img:train_batch1}~и~\ref{img:train_batch2} представлены примеры использования описанных выше оптимизаций в процессе обучения.

\includeimage{train_batch1}{f}{h}{0.9\textwidth}{Батч обучающей выборки с применением мозаичной аугментации. Часть 1}

\includeimage{train_batch2}{f}{h}{0.9\textwidth}{Батч обучающей выборки с применением мозаичной аугментации. Часть 2}

На рисунке~\ref{img:val_batch1_pred} представлен пример использования описанных выше оптимизаций в процессе валидации результатов работы обученной модели.

\includeimage{val_batch1_pred}{f}{h}{0.9\textwidth}{Батч валидационной выборки}

\section{Результаты обучения слабых экспертов}

\subsection*{Kaggle Ships In Google Earth}

\includelisting{train_1.txt}{Результат обучения слабого эксперта}

\begin{figure}[htp]
	\centering
	\begin{tikzpicture}
		\begin{axis}[
			axis lines=left,
			xlabel={Эпоха},
			ylabel={P},
			legend pos=north west,
			ymajorgrids=true,
			xmajorgrids=true,
			xtick={10,20,30,40,50,60,70,80,90,100},	
			ytick={0.7,0.72,0.74,0.76,0.78,0.8,0.82,0.84,0.86,0.88,0.90,0.92,0.94,0.96}
		]
			\addplot table[x=epoch,y=metrics/precision(B),col sep=comma] {inc/csv/train_1.csv};
            %\legend{Без кэширования, С кэшированием}
		\end{axis}
	\end{tikzpicture}
	\captionsetup{justification=centering}
	\caption{Точность слабых экспертов (P)}
	\label{plt:precision}
\end{figure}

\subsection*{ShipRSImageNet}

\includelisting{train_2.txt}{Результат обучения слабого эксперта}

\begin{figure}[htp]
	\centering
	\begin{tikzpicture}
		\begin{axis}[
			axis lines=left,
			xlabel={Эпоха},
			ylabel={P},
			legend pos=north west,
			ymajorgrids=true,
			xmajorgrids=true,
			xtick={10,20,30,40,50,60,70,80,90,100},	
			%ytick={0.7,0.72,0.74,0.76,0.78,0.8,0.82,0.84,0.86,0.88,0.90,0.92,0.94,0.96}
		]
			\addplot table[x=epoch,y=metrics/precision(B),col sep=comma] {inc/csv/train_2.csv};
            %\legend{Без кэширования, С кэшированием}
		\end{axis}
	\end{tikzpicture}
	\captionsetup{justification=centering}
	\caption{Точность слабых экспертов (P)}
	\label{plt:precision_2}
\end{figure}

\subsection*{Huawei Ship}

\includelisting{train_3.txt}{Результат обучения слабого эксперта}

\begin{figure}[htp]
	\centering
	\begin{tikzpicture}
		\begin{axis}[
			axis lines=left,
			xlabel={Эпоха},
			ylabel={P},
			legend pos=north west,
			ymajorgrids=true,
			xmajorgrids=true,
%			xtick={10,20,30,40,50,60,70,80,90,100},	
			%ytick={0.7,0.72,0.74,0.76,0.78,0.8,0.82,0.84,0.86,0.88,0.90,0.92,0.94,0.96}
		]
			\addplot table[x=epoch,y=metrics/precision(B),col sep=comma] {inc/csv/train_3.csv};
            %\legend{Без кэширования, С кэшированием}
		\end{axis}
	\end{tikzpicture}
	\captionsetup{justification=centering}
	\caption{Точность слабых экспертов (P)}
	\label{plt:precision_3}
\end{figure}

\section{Примеры использования разработанного программного комплекса}

\section{Вывод}