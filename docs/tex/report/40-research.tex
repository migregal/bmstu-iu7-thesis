\chapter{Исследовательский раздел}

\section{Аппаратное обеспечение}

Исследования разработанного метода, а так же реализованного программного комплекса, будут производиться под управлением операционной системы Ubuntu~22.04~via~WSL~2~on~Windows~10 с использованием следующего аппаратного обеспечения:
\begin{itemize}[label=---]
    \item CPU: Intel~Core™~i7-4790~CPU~@~3.60ГГц;
    \item GPU: NVIDIA~GeForce~RTX~2060~6144Мб;
    \item RAM: 16~Гб.
\end{itemize}

Как отмечалось ранее, используемый GPU поддерживает технологию CUDA версии 12.0, являюшуюся последней опубликованной версией на момент написания работы.

Указанная в данном пункте информация должна быть учтена при оценивании результатов работы разработанного программного обеспечения и, в частности, при оценке времени отклика системы.

\section{Влияние размера входного изображения}

На рисунке~\ref{plt:steps-timing} представлена полученная экспериментально зависимость времени этапов обработки изображения от размера последнего.

\begin{figure}[htp]
	\centering
	\begin{tikzpicture}
		\begin{axis}[
			width=.5\textwidth,
			height=.4\textwidth,
			axis lines=left,
			xlabel={Размерность, пикс.},
			ylabel={T,~мс},
			legend style={at={(axis cs:784,18)},anchor=south west},
			ymajorgrids=true,
			ytick={0,10,20,30,40,50,60,70,80,90,100},
			xmajorgrids=true,
			xticklabel style={rotate=45,anchor=east},
			xtick={640,784,880,1280},
		]
			\addplot table[x=size,y=preprocess,col sep=comma, mark=halfdiamond*] {inc/csv/time_parallel.csv};
			\addplot table[x=size,y=inference,col sep=comma] {inc/csv/time_parallel.csv};
			\addplot table[x=size,y=postprocess,col sep=comma] {inc/csv/time_parallel.csv};
            \legend{Предобработка, Обработка, Постобработка}
		\end{axis}
	\end{tikzpicture}
	\captionsetup{justification=centering}
	\caption{Время работы в зависимости от размерности (T)}
	\label{plt:steps-timing}
\end{figure}

Отметим, что в разработанном методе используются нейронные сети, т.е. время обработки изображения должно зависеть только от стадий предобработки изображения и постобработки результатов, чем и объясняется практически константное время обработки изображения.

На рисунке~\ref{plt:total-timing} предствалена зависимость полного времени обработки от размерности. Представлены экспериментальные данные для двух способов проведения вычислений: при вычислении в удаленном кластере и локально, на клиентской машине.

\begin{figure}[htp]
	\centering
	\begin{tikzpicture}
		\begin{axis}[
			width=.5\textwidth,
			height=.4\textwidth,
			axis lines=left,
			xlabel={Размерность, пикс.},
			ylabel={T,~с},
			legend pos=south east,
			ymajorgrids=true,
			xmajorgrids=true,
			xticklabel style={rotate=45,anchor=east},
			xtick={640,784,880,1280},
		]
			\addplot table[x=size,y=total,col sep=comma] {inc/csv/time_parallel.csv};
			\addplot table[x=size,y=total_remote,col sep=comma] {inc/csv/time_parallel.csv};
			\legend{Локальные, Кластерные}
		\end{axis}
	\end{tikzpicture}
	\captionsetup{justification=centering}
	\caption{Время работы в зависимости от размерности (T)}
	\label{plt:total-timing}
\end{figure}

Заметная разница во времени работы между локальными и кластерными вычислениями объясняется несколькими причинами.

В первую очередь повлияла конфигурация самого кластера --- в виду ограниченности ресурсов в ходе исследования, кластер был развернут на одной машине. Так как в архитектуре Ray мастер-узел кластера берет на себе нагрузку по управлению ресурсами кластера, вычислительные можности были существенно урезаны.

Кроме того, значительное влияние оказывает способ использования кластера. Разработчики Ray рекомендуют предварительно загружать большие объекты в кластер, чтобы избежать излишних затрат на многократную передачу последних по сети, а так же сопутствующих сериализации и десериализации. Под большими объектами в данном констексте понимаются программные сущности занимающие 1~Мб и более. Тем не менее, текущая реализация клиента подразумевает загрузку моделей нейронной сети, а так же обрабатываемого изображения при каждом запуске. Для решения этой проблемы следует перейти на клиент--серверную архитектуру, где сервер будет единократно загружать необходимые объекты в кластер, что должно значительно снизить время работы.

\section{Влияние порогового значения}

На рисунке~\ref{plt:limit-impact} представлена полученная экспериментально зависимость точности и полноты распознавания от порогового значения.

Как указывалось нарее, пороговое значение используется в алгоритме объединения результатов работы слабых экспертов для обнаружения пересекающихся обрамляющих окон.

\begin{figure}[htp]
	\centering
	\begin{tikzpicture}
		\begin{axis}[
			width=.5\textwidth,
			height=.4\textwidth,
			axis lines=left,
			xlabel={Пороговое значение},
			ylabel={P},
			legend pos=south east,
			ymajorgrids=true,
			xmajorgrids=true,
			xticklabel style={rotate=45,anchor=east},
			xtick={0.5,0.6,0.75,0.85,0.9},
		]
			\addplot table[x=limit,y=precision,col sep=comma] {inc/csv/limit.csv};
			\addplot table[x=limit,y=recall,col sep=comma] {inc/csv/limit.csv};
            \legend{Точноть, Полнота}
		\end{axis}
	\end{tikzpicture}
	\captionsetup{justification=centering}
		\caption{Точность (P) и полнота (R) в зависимости от порогового значения.}
	\label{plt:limit-impact}
\end{figure}

Стоит отметить, что при пороговых значениях не ниже $0.75$ пркатически отсутствует разница в точности разспознавания, однако продолжает расти показатель полноты. 

Такие результаты вполне объяснимы --- точность, характеризующая отношение числа <<истинных>> объектов к общему числу распознанных объектов, перестает расти после определенного порога, так как все--таки присутствует систематическая ошибка в результатах работы слабых экспертов. 

В то же время, полнота, характеризующая отношение успешных распознаваний к действительному числу объектов, растет в связи с особенностью спользуемых наборов данных --- снимки, на которых распознаваемые объекты имели бы большие пересечения (более 70~\% площади), представлены в относительно малых количествах, в связи с чем использование более высоких пороговых значений ведет к снижению ложных срабатываний метода.

\section{Вывод}

Была проведена оценика результаты работы метода в зависимости от различных параметров системы.

У разработанного программного комплекса можно выявить следующие достоинства и недостатки.
Достоинства:
\begin{itemize}[label=---]
    \item универсальность --- возможно использование при анализе снимков в разных проекциях, разных размеров и разных форматах.
    \item высокая точность --- отсутсвие потери точности при разных размерах исходного снимка;
    \item малое время ответа --- проведение параллельных вычислений в отдельном кластере.
\end{itemize}

Недостатки:
\begin{itemize}[label=---]
    \item использование кластера --- для того, чтобы получить выигрыш по времени ответа, в сравнении с локальными вычислениями, необходимо не менее двух узлов кластера, а так же хотя бы один GPU.
\end{itemize}